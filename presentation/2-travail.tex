%!TEX root = presentation.tex
\section{Présentation du travail}%FIXME

% \begin{frame}{Stats}
% %%LOCK PACKGS...
% \end{frame}
\subsection{Architecture}

\begin{frame}{Architecture Globale}

\only<+>{
\begin{block}{}
    \begin{itemize}
        \item LoginActivity
        \item MainActivity
    \end{itemize}
\end{block}
}

\only<+>{
\begin{figure}
\center
\pgfuseimage{mvp}
\caption{Modéle-View-Controller: Passive View}
\end{figure}
}

\only<+>{
\begin{figure}
\center
\pgfuseimage{cls_global}
\caption{Diagramme de classe UML de MainActivity}
\end{figure}
}

\end{frame}

\section{Localisation}
\begin{frame}{Localisation}

\only<+>{
\begin{block}{Critéres}
    \begin{itemize}
        \item Provider.
        \item Power.
        \item Min Time.
        \item Min Distance.
    \end{itemize}
\end{block}
}

\end{frame}

\section{Threading}
\begin{frame}{Threading} %FIXME
% Consideration pour une application android.
\only<+>{
\begin{figure}
\centering
\pgfuseimage{anr}
\end{figure}
    \begin{itemize}
        \item ANR (Application Not Responding).
        \item Pas de réponse au événement d'entrée (touches et/ou taps) pendants 5 secondes ou un \dev{BroadcastReceiver} exécuter pendant plus de 10s.
        \item  Par défaut, les apps android fonctionne dans un thread unique...
    \end{itemize}
    %%%SPEECH en tant que developpeur android, non selement vous
    %%%developpez des applications, mais aussi des reflexe et des manies
    %%%s "esque ca va block la si j'effectue une transaction ici..."
}

\only<+>{
\begin{block}{Techniques}
    \begin{itemize}
        \item AsyncTask.
        \item Handler.
        \item ScheduledExecutorService.
    \end{itemize}
\end{block}
}

\end{frame}

\section{Conscience de l'état du terminal}
\begin{frame}{Conscience de l'état du terminal}
\only<+>{
    \begin{itemize}
        \item Connectivité.
        \item En charge ou pas.
        \item Battery.
    \end{itemize}
}

%Speech: Faudra fair attention ici, notre but n'ai pas d'assuré l'authentification, c'est le travail
% de la class final qui implémente cette interface. L'idée ici est de passé l'information - sous entendu
% l'identifiant et le mot de pass - à la class spécialisé dans les meilleurs conditions.

\end{frame}