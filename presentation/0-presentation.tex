\documentclass[
ucs
% ,13pt
, xcolor=table
% , handout
]{beamer}

\mode<presentation>{
	\usetheme{Madrid}
	\setbeamercovered{transparent}
}

\usepackage[utf8x]{inputenc}
\usepackage[OT1]{fontenc}
\usepackage[frenchb]{babel}
\usefonttheme{serif}

\usepackage{hyperref}
\hypersetup{
    pdfauthor={Habbachi Riadh},
    pdfsubject={Présentation de projet de fin d'étude cycle ingénieur},
    pdftitle={Présentation de projet de fin d'étude: Développement d'une application Android pour médecins},
    pdfkeywords={ENIG, Stage, ingénieur, LaTeX, PDF, hyper-links}
}

%\usepackage{time}

\usepackage{graphicx}

%\usepackage{subfigure}

\title[Projet de fin d'etudes]
{Développement d'une application Android pour médecins}
\author{Riadh Habbachi}
\institute[ENIG]{Ecole Nationale d'Ingénieurs de Gabes}
\date[]{\today}
\subject{}

\AtBeginSubsection[]{
	\begin{frame}<beamer>{Outline}
	\tableofcontents[currentsection, currentsubsection]
	\end{frame}
}

\makeindex

% \pgfdeclareimage[height=0.1\textheight]{tunav-logo}{res/drawable/tunav-logo.png}
% \logo{\pgfuseimage{tunav-logo}}

\let\Tiny=\tiny% Gets rid of Font shape `OT1/cmss/m/n' warning
\begin{document}%%%%%%

\begin{frame}
	\titlepage
\end{frame}

\begin{frame}{Plan}
	\tableofcontents
\end{frame}

\section{Introduction}
\subsection{Pourquoi Android?}

\begin{frame}{Généralités}%FIXME
%%% HISTORIQUE
\only<+>{
	
}
%%% MARKET SHARE AND POPULARITY
\only<+>{
	
}
\end{frame}

\begin{frame}{Développement dans un environnement mobile Android} %FIXME
%%% 
%Régle de si un processus est irrosponsive pandants 5s...
\end{frame}

\subsection{Pourquoi TunavMédi?}%FIXME
\begin{frame}{Objectifs}

\end{frame}

\begin{frame}{Scénarios de déploiement}
%%% MAIN SECTION
%%% SCENARIO 1
\only<+>{
	
}

%%% SCENARIO 2
\only<+>{
	
}
\end{frame}

\section{Présentation du travail}%FIXME

\subsection{Point de vu architectural}
\begin{frame}{Diagramme de packages}

\end{frame}

\subsection{Point de vu design \& ergonomie}

%%% LAST SECTION
\section{Conclusion \& Perspectives}

\begin{frame}{Conclusion}
%%% THE GOOD
\only<+>{
	\begin{block}{Ce qu'on a aimé}
	
	\end{block}
}

%%% THE BAD
\only{
	\begin{block}{Ce qu'on peut amélioré}
	
	\end{block}
}

%%% THE UGLY
\only{
	\begin{block}{Ce qu'on aurai dû faire}
	
	\end{block}
}
\end{frame}

\begin{frame}{Perspectives}
\only<+>{
	\centerline{4 mois c'est un peut court pour un projet...}
	%Speech: ce qui peut en parti justifié les travaux médiocre présenté par les éleves...
}
\only<+>{
	\begin{itemize}
	\item Perspectives immédiat.
	\item Perspectives future.
	\end{itemize}
}
\end{frame}

\begin{frame}{Perspectives Immédiat}

\end{frame}

\begin{frame}{Perspectives Future}
\only<+>{
	Intégration à un (des) système(s) d'Information de gestion des hôpitaux et services de santé.
	\begin{itemize}
	\item GNU Health.
	\end{itemize}
}
\only<+>{
	Système d'identification des patients admit
	%%%Speech
	% Un des goulot pour les systèmes de santé est l'admission des 
	% nouveaux patients. Particulièrement le comment notifier les parties concerné par l'arrivé de ce patient dans l’établissement, exemple si le malade à besoin d'un scan RMI, il devrait étre automatiquement reconnu quant il se présente devant le service en question.
	\begin{itemize}
	\item NFC?
	\item QRcode?
	\end{itemize}
}
\end{frame}

\section*{Références \& Remerciement}
\begin{frame}
%%%Speech
%Donc voyons que vous êtes encore là, et éveiller en plus! ^_^...
\centerline{Merci!}
\end{frame}

\end{document}