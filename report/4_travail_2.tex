%!TEX root = report.tex

\chapter{Étude Structurelle}

\section{Introduction du chapitre}

Dans ce chapitre on présente la structure globale de l'application

\section[Couche d'Accès aux Données]{Couche D'Accés aux Données}

Un des objectifs principals de ce projet étant de fournir une solution
d’accès aux données flexibles afin de couvrir les besoins de chaque
client de manière individuelle. On a opté donc pour un modèle basé sur
l’implémentation de deux interfaces (figure \ref{fig:cls_dal}):

\begin{itemize}

\item Interface d'authentification.

\item Interface d’accès à la liste des patients.

\end{itemize}

\begin{figure}
\center
\includegraphics[totalheight=0.5\textheight]{diagrams/cls_dal}
\caption{Diagramme de classes \gls{uml} des interfaces de la couche d’accès.}
\label{fig:cls_dal}
\end{figure}

L'idée est simple: pour chaque client, une implémentation spécifique à son infrastructure sera développée soit par son propre effectif, soit par une des équipes de \textsc{Tunav}, ou dans le cas idéal par une alliance formé par des agents des deux camps qui garantit une collaboration plus poussée pour des résultats meilleurs.
Ces ensembles d'interfaces nous permettent de construire notre application.

\subsection{Interface d'authentification}

\dev{AuthenticationHandler}\footnotetext{com.tunav.tunavmedi.dal.abstraction.AuthenticationHandler} (figure
\ref{fig:cls_dal}) est une classe abstraite comportant les méthodes
requise par notre application pour effectuer les actions
d'authentification, de dé-authentification, de vérification
d'authenticité ainsi l'obtention des informations associées à l'utilisateur
authentifié.

Malgré la variété des techniques d'authentification utilisée dans le domaine
informatique, l'étape d'acquisition des identificateurs de l'utilisateur
représente un point de départ commun. On utilise ce caractère dans l'interface
d'authentification en demandant à nos clients d'implémenter la méthode
\dev{login()} qui prend en argument l'identifiant et le mot de passe fourni par
le médecin, dans le cas d'une éventuelle erreur d'authentification,
l’implémentation met a notre disposition un message d'erreur accessible par la
méthode \dev{getError()}. Pour effectuer l’opération inverse le client
implémente la méthode \dev{logout()} supposée annoncée au service distant la dé-
authentification de l'utilisateur du terminal. Pour vérifier le l'état actuel de
la relation du terminal avec la base distante, on utilise le booléen retourné
par \dev{getStatus()}, utile dans les cas de déconnexion temporaire ou du
redémarrage de notre application. Les méthodes \dev{getDisplayName()} et
\dev{getPhoto()} retournent respectivement le nom de l'utilisateur et sa photo.

\subsection{Interface d’accès à la liste des patients}

\dev{PatientsHandler}\footnotetext{com.tunav.tunavmedi.dal.abstraction.PatientsH
andler} (figure \ref{fig:cls_dal}) est une classe abstraite comportant les
méthodes requises par notre application pour effectuer les actions de mise à
jour de la liste des patients dans le deux sens (terminal $\rightarrow$ service
et terminal $\leftarrow$ service), elle contient aussi un objet de type
\dev{Runnable}\footnotetext{java.lang.Runnable} associé au mécanisme de notification.

\subsubsection{Mécanisme de notification}

Le patron \textbf{Observateur} (\en{observer pattern}) (fig
\ref{fig:observer}) et un patron de conception couramment utilisé et qui
nous permet d'avoir une relation 1$\rightarrow$N entre divers objets. Le
patron observateur assume que l'objet qui contient les données est
séparé des objets qui les affiche et ces dites objets observent le
changement de ces données~\cite{jdp_observer}. Quand on implémente le
patron observateur, on réfère communément à l'objet contenant les
données par "Sujet"; et chacun des consommateurs des données par
"Observateur", Et chaque Observateur implémente une interface préconçue
que le Sujet invoque quant les données changent~\cite{jdp_observer}.

\begin{figure}[H]
\center
\includegraphics[width=0.8\textwidth]{Observer}
\caption{Diagramme \gls{uml} du patron de conception Observateur~\cite{wikipedia:observer}}
\label{fig:observer}
\end{figure}

Dans le langage Java, ce patron est réalisé à travers la classe \dev{java.util.Observable} et l'interface \dev{Observer}\footnote{Java.util.Observer}. Le Sujet hérite de la classe \dev{Observable} et les changements sont signalés par les méthodes \dev{setChanged()} et \dev{notifyObservers()} ou \dev{notifyObservers(Object message)}.

\subsubsection{Les objets de données}

\begin{figure}
\center
\includegraphics[width=0.8\textwidth]{diagrams/cls_dal_patient}
\caption{Diagramme de la classe \gls{uml} \dev{Patient} }
\label{fig:cls_dal_patient}
\end{figure}

La communication des données avec le service distant se fait à travers l'objet \dev{Patient}\footnote{com.tunav.tunavmedi.dal.datatype.Patient} (figure \ref{fig:cls_dal_patient}. Cet objet contient toutes les informations requises pour la synchronisation et l'affichage et la gestion des patients, en particulier le dossier médical et la position actuelle du patient.

\paragraph{Synchronisation:}

Étant sujet aux modifications de la part de l'application et du serveur distant, un problème se pose pour connaître la version la plus à jour. Pour cela chaque modification apportée est suivi par une mise à jour de variable \dev{mUpdated} par le temps à cette instance précise (cette opération est interne à l'objet). En cas oú deux versions différentes de l'objet \dev{Patient} avec le même \dev{mID} et \dev{mUpdated}, le service est supposé favoriser sa version.

\paragraph{Dossier médical:}

Le dossier médical est fournit sous le format \en{HTML}. Cette représentation est idéale car elle nous permet de faire abstraction sur le format du dossier tel que sauvegardé par l'établissement client, la couche d’accès assurant éventuellement la conversion.


\paragraph{Position:}

La position actuelle du patient est représentée par un objet \dev{Placemark}\footnote{com.tunav.tunavmedi.dal.datatype.Placemark} inspiré par la notation XML \gls{kml}. Les coordonnées sont représentées par un objet de type \dev{Location}\footnote{android.location.Location}.

\subsection{Implémentation de tests}\label{subsection:dal_impl}

Le package \dev{com.tunav.tunavmedi.dal.sqlite} contiens une
implémentation de la couche d’accès abstraite (figure
\ref{fig:dal_sqlite}) réalisée dans le cadre de ce projet pour
pouvoir tester la solution. Cette implémentation est de caractère local
à l'application à travers les \gls{api} de la base de données
\dev{SQLite} qui fait partie de l'\gls{sdk} \android{}. En fait une
implémentation locale nous affranchie des problèmes qui peuvent se
produire et dont la corrélation avec l'application est faible. Cette même
idée a influencé la mise en place même de cette implémentation qui à su
rester la plus simple possible en restant très proches des objets de
base de notre application.

\begin{figure}
\center
\includegraphics[angle=-90, width=0.8\textwidth]{diagrams/cls_dal_sqlite}
\caption{Diagramme de classe de l'implémentation de la couche d'accès de tests  à base de SQLite.}
\label{fig:dal_sqlite}
\end{figure}

Cette implémentation peut être subdivisée en trois éléments: Les \dev{Contrats}, les \dev{Helpers}, et la classe \dev{DBSetup}.

\paragraph{Les Contrats:}

Représente les contrats relatifs aux tables dans notre implémentation de
tests. Chaque contrat implémente l'interface
\dev{BaseColumns}\footnote{android.provider.BaseColumns} et contient - entre autre - les
commandes SQL de création et de suppression de la dite table, des
éventuels index, et les commande d'insertion des données de test.

\paragraph{Les \en{Helpers}:} 

Ce sont les implémentations des classes abstraites qui définissent la
couche d’accès et présentent les procédures d'extraction des données pré-
insérées dans nos tables fictives en faisant appel à la classe
\dev{DBSetup} .

\paragraph[La classe \dev{DBSetup}:]{La classe \dev{DBSetup}:} 

Elle hérite de la classe \dev{SQLiteOpenHelper} et est destinée à
contrôler la création et l’accès à notre base de données de tests.

\section{Structure de l'Application}

L'application est subdivisée en deux parties majeures représentés par deux classes de type \dev{Activity}\footnotetext{android.app.Activity} :

\begin{description}

\item [LoginActivity:] C'est une entité indépendante qui implémente la logique d'authentification, 

\item [MainActivity:] C'est l'entité principale de notre solution mobile, elle relie les divers composants utilisés dans la transmission de la liste des patients, de la localisation et de la Conscience de l'état du terminal.

\end{description}

\subsection{LoginActivity}

La \dev{LoginActivity} s'occupe de la partie d'authentification et de dé-authentification. Elle utilise un \dev{AsyncTask}\footnote{android.os.AsyncTask} pour effectuer ces opérations de manière non bloquante.

\begin{lstlisting}[language=xml, caption=Déclaration de LoginActivity dans AndroidManifest]

        <activity
            android:name="com.tunav.tunavmedi.activity.LoginActivity"
            android:label="@string/app_name"
            android:screenOrientation="portrait" >
            <intent-filter>
                <category android:name="android.intent.category.DEFAULT" />

                <action android:name="com.tunav.tunavmedi.action.LOGOUT" />
                <action android:name="com.tunav.tunavmedi.action.LOGIN" />
            </intent-filter>
        </activity>

\end{lstlisting}



\subsection{MainActivity}

\begin{lstlisting}[language=xml, caption=Déclaration dans AndroidManifest de MainActivity]

        <activity
            android:name="com.tunav.tunavmedi.activity.MainActivity"
            android:label="@string/title_activity_main"
            android:screenOrientation="portrait" >
            <intent-filter>
                <action android:name="android.intent.action.MAIN" />

                <category android:name="android.intent.category.DEFAULT" />
                <category android:name="android.intent.category.LAUNCHER" />
            </intent-filter>
        </activity>

\end{lstlisting}

L'architecture globale du \dev{MainActivity}  (figure \ref{fig:cls_global})
est calquée sur Le patron "Vue Passive" (Passive View Pattern). Le patron
\en{Passive View} (fig \ref{fig:passive_view}) est une variation des
patrons \gls{mvc} et \gls{mvp}, de ce qui ce passe dans ces patrons.

L'interface utilisateur est divisée entre une Vue qui s'occupe de
l'affichage des données et un contrôleur qui répond aux interactions de
l'utilisateur. La différence majeur avec le \en{Passive View} est que la
Vue est complètement passive et n'est pas responsable de sa mise à jour
depuis le modèle. Dans ce cas toute la logique de la Vue est dans le
contrôleur et aucune dépendance ni dans un sens au dans un autre entre
la Vue et le modèle~\cite{fowler:passive_view}.

\begin{figure}
\center
\includegraphics[width=0.6\textwidth]{passive_view}
\caption{Diagramme \gls{uml} de composants du patron \en{Passive View}~\cite{fowler:passive_view}}
\label{fig:passive_view}
\end{figure}

Ce patron est idéal dans notre cas pour deux raisons majeures:

\begin{itemize} 

\item Dans notre projet la Vue n'est pas la partie la plus importante
dans la mesure où l'objectif est d'intégrer un système développé
parallèlement, donc éventuellement avec une autre logique de
présentation. Déporter les interactions avec le modèle dans le
contrôleur permet d'intégrer d'autres implémentations d'affichage plus
facilement.

\item La nature même de cette procédure d’accès - à savoir l’aspect
abstrait, donc plus fragile - nous conduit à réduire les composants en
relations pour réduire la marge d'erreur possible et facilité les tests d’intégration.

\end{itemize}

\begin{figure}
\center
\includegraphics[angle=-90, width=0.9\textwidth]{diagrams/cls_global}
\caption{Diagramme de classes \gls{uml} de l'architecture générale de l'application.}
\label{fig:cls_global}
\end{figure}

Dans la suite de ce chapitre, on procède à l'explication détaillée du Contrôleur
et de la Vue, pour le Modèle se vous reporter au point
\ref{subsection:dal_impl}.

\subsubsection{Le Contrôleur}

\begin{figure}
\center
\includegraphics[width=0.8\textwidth]{diagrams/cls_ctrl}
\caption{Diagramme \gls{uml} de classe du contrôleur.}
\label{fig:cls_ctrl}
\end{figure}

\begin{lstlisting}[language=xml, caption=Déclaration dans AndroidManifest du PatientService]

        <service
            android:name="com.tunav.tunavmedi.service.PatientsService"
            android:exported="false" >
            <intent-filter>
                <category android:name="android.intent.category.DEFAULT" />
            </intent-filter>
        </service>

\end{lstlisting}

Notre contrôleur est matérialisé par l'objet
\dev{PatientService}\footnote{com.tunav.tunavmedi.service.PatientService} qui
hérite de la classe \dev{Service}\footnote{android.app.Service}. L'\gls{api}
\android{} définie un service comme étant un composant de l'application qui
représente soit la volonté de cette application de faire des longues opérations
sans interagir avec l'utilisateur ou d’offrir des fonctionnalités à
l'intention des autres applications\cite{api:service}.

\paragraph{Localisation:}
Pour la localisation, le contrôleur implémente les techniques présentées dans \ref{sss:android_localisation}
%TODO

\paragraph{Conscience de l'état du terminal}

Le tableau \ref{tab:status} représente la configuration que le contrôleur suit dans le cas d'un changement d’état du terminal. En particulier l'inutilité de chercher la position actuelle du terminal dans le cas ou celui ci est en charge, encore en cas ou le terminal annonce que la batterie est faible, on procède à un mode d’économie d’énergie pour éviter entre autres la corruption des données.

\begin{table}[H]
\centering
\begin{tabular}{|c|c|c|}
\hline
&\textsf{Mises à jour} & \textsf{Localisation}\\
\hline
\textsf{Batterie Faible} & déactivé & déactivé\\
\hline
\textsf{Batterie en Charge} & activé & déactivé\\
\hline
\textsf{Pas de connectivité} & déactivé & activé\\
\hline
\end{tabular}
\caption{Configuration du contrôleur en réponse au changement d’état du terminal}
\label{tab:status}
\end{table}

\subparagraph{Connectivité:}

Les permissions citées dans le listing \ref{lst:permission_network} sont nécessaires pour s’abonner aux événement liés à la connectivité du terminal. Le \dev{NetworkReceiver}\footnote{com.tunav.tunavmedi.broadcastreceiver.NetworkReceiver} (listing \ref{lst:receiver_network}) s'occupe de notifier les intéressés, dans notre cas le contrôleur, via des \dev{SharedPreferences}.

\begin{lstlisting}[language=xml, label=lst:permission_network, caption=Déclaration dans AndroidManifest des permissions d’accès à l’état des interfaces réseaux.]

<uses-permission android:name="android.permission.ACCESS_NETWORK_STATE" />

\end{lstlisting}

\begin{lstlisting}[language=xml, label=lst:receiver_network, caption=Déclaration dans AndroidManifest du  NetworkReceiver]

        <receiver
            android:name="com.tunav.tunavmedi.broadcastreceiver.NetworkReceiver"
            android:enabled="true" >
            <intent-filter>
                <action android:name="ConnectivityManager.CONNECTIVITY_ACTION" />
            </intent-filter>
        </receiver>

\end{lstlisting}

\subparagraph{Batterie}

Le \dev{BatteryReceiver}\footnote{com.tunav.tunavmedi.broadcastreceiver.BatteryReceiver} (listing \ref{lst:receiver_battery}) fait savoir au contrôleur via des \dev{SharedPreferences} si la batterie est faible ou pas.

\begin{lstlisting}[language=xml, label=lst:receiver_battery, caption=Déclaration dans AndroidManifest du BatteryReceiver.]

        <receiver
            android:name="com.tunav.tunavmedi.broadcastreceiver.BatteryReceiver"
            android:enabled="true" >
            <intent-filter>
                <action android:name="android.intent.action.BATTERY_LOW" />
                <action android:name="android.intent.action.BATTERY_OK" />
            </intent-filter>
        </receiver>

\end{lstlisting}

\subparagraph{Mobilité:}
Le \dev{ChargingReceiver}\footnote{com.tunav.tunavmedi.broadcastreceiver.ChargingReceiver} (listing \ref{lst:receiver_network}) est destiné à notifier le contrôleur quand le terminal est connecté ou déconnecté du  chargeur via des \dev{SharedPreferences}.

\begin{lstlisting}[language=xml, label=lst:receiver_charging, caption=Déclaration dans AndroidManifest ChargingReceivers.]

        <receiver
            android:name="com.tunav.tunavmedi.broadcastreceiver.ChargingReceiver"
            android:enabled="true" >
            <intent-filter>
                <action android:name="android.intent.action.ACTION_POWER_CONNECTED" />
                <action android:name="android.intent.action.ACTION_POWER_DISCONNECTED" />
            </intent-filter>
        </receiver>

\end{lstlisting}

\subsubsection{La Vue}

\begin{figure}
\center
\includegraphics[width=0.8\textwidth]{diagrams/cls_view}
\end{figure}

Le système d'exploitation \android{} rend facile le développement des applications qui tournent sur des appareils qui possèdent des formes et des tailles d’écrans différentes, une des améliorations apportées dans \android{} 3.0 Honeycomb sont les \dev{Fragment} censé décomposé les fonctionnalité et les interfaces utilisateur d'une l'application \android{} en des modules réutilisables. Notre implémentation de la Vue prend avantage de cette introduction en utilisant des \dev{Fragments} et se compose essentiellement de 4 composants:

\begin{description}

\item[PatientAdapter] Représente un \dev{BaseAdapter} qui joue le rôle d'un adaptateur entre la \dev{PatientListFragment} et notre contrôleur, la communication avec celui-ci est assuré à travers l'interface \dev{PatientsListener}.

\item[PatientListFragment] Hérite de l'objet \dev{PatientListFragment} et s'occupe de l'affichage de la liste des patients.

\item[PatientDisplay] Un \dev{DialogFragment} qui s'occupe de l'affichage du dossier médical du patient

\item[PatientOptions] Un \dev{DialogFragment} qui permet au docteur de modifier la condition d'un patient.

\end{description}

\paragraph{Algorithme de Trie}

L'or de l'affichage de la liste des patients, une opération de trie est appliquée pour faciliter la tache du docteur en mettant en valeur les cas qui requièrent le plus son attention.
L’algorithme se base sur les conditions suivantes (dans l'ordre):

\begin{enumerate}

\item Le patient est un cas urgent ou non.

\item Le patient est à proximité ou non.

\item La date d'admission du patient.

\end{enumerate}

Cette opération est réalisé la méthode static \dev{java.util.Collections.sort()} en utilisant notre propre objet de type \dev{java.util.Comparator<Patient>} qui respecte les conditions citées ci-dessus.


\paragraph{Affichage du dossier médical}

Le dossier étant sous la forme d'un document HTML, pour l'afficher on utilise un \dev{WebView}\footnote{android.webkit.WebView} (listing \ref{lst:xml_patientdisplay})qui nous offre les capacités d'affichage d'un vrai navigateur web.


\begin{lstlisting}[language=xml, label=lst:xml_patientdisplay, caption=Déclaration XML du WebView utilisé par PatientDisplay]

    <WebView
        android:id="@+id/task_dialog_description"
        style="@style/ListDescription"
        android:layout_width="match_parent"
        android:layout_height="wrap_content"
        android:layout_below="@id/task_dialog_separator"
        android:textIsSelectable="true" 
        android:singleLine="false"/>

\end{lstlisting}