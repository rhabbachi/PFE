% !TEX root = ../00-main.tex

\chapter{Réalisation}
\section{Préparation de l'environnement de développement}%TODO
\section{Testes}
\subsection{Pourquoi tester?~\cite{pycon:getting_started_with_automated_testing}}
\begin{itemize}
\item La raison la plus évidente pour écrire les testes - étant la plus populaires aussi - est que c'est un moyen efficace pour savoir si le bout de code ajouté dans le projet marche correctement ou pas, ce qui non seulement fournit une certaine confiance dans la robustesse du logiciel mais présente un effet secondaire bénéfique en réduisant le temps nécessaire pour le débogage à la recherche d'un bug caché. %FIXME
\item Une autre raison pour écrire les testes est que c'est un autre forme de documentation qui aide les autres développeurs comprendre le code contenu dans le projet.
\item Une raison non évidente pour écrire des teste est le fait que les tests améliore la manière dont le code est conçu en mettant en évidence les difficultés pour le maintenir.
\end{itemize}
\subsection{Modifié la localisation dans l'émulateur}%FIXME