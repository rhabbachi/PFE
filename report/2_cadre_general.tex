%!TEX root = report.tex

\chapter{Cadre Général du Projet}

\section{Introduction du Chapitre}

Ce chapitre est subdivisé en deux parties: la première partie est
consacrée à la présentation de l’organisme d’accueil \textbf{TUNAV}. La
deuxième partie est destinée à la présentation du projet en soit.

\section{Présentation de l'organisme d'accueil}  

\begin{figure}
\center
\includegraphics[scale=1]{tunav-logo}
\caption{Logo \textsc{Tunav}.}
\end{figure}

TUNAV se situe à la Cité Technologique des Communications, Parc
Technologique El Gazala à l’ARIANA, et a été fondée par son Président
Directeur Général Mohamed Anis Kallel.

En guise de présentation, rien de mieux que de l’avoir directement du patron lui-même\cite{index_tunisie}:

"\textsc{Tunav} est une société technologique, créée au mois d’août
2004, implantée à la technopole El Gazala et spécialisée dans la
technologie GPS et ses diverses applications dans les domaines de
navigation et de gestion de flotte."

"\textsc{Tunav} est connue en Tunisie par son système \og{}LaTrace\fg{}
de gestion de flotte par GPS, lequel a été commercialisée pour la
première fois en Octobre 2005. Il s'agit d'un système articulé autour
d'une application très évoluée de gestion de flotte, d'une gamme
d'appareils GPS/GPRS et d'une base de données géographique richement
renseignée."

\textsc{Tunav} possède un savoir faire reconnu dans le domaine de la
localisation qui peut être exploité dans le domaine médical.

\section{Présentation du projet}

Ce projet s'inscrit dans un effort pour faciliter le travail des médecins en
leurs rapprochant de leur patient à travers des technologies de localisation.
chaque médecin authentifier a accès à la liste de ses patients ordonner dans
l'ordre de leur cas (urgent ou non), leur proximité géographique dans
l’établissement, et leur date d’admission dans l'hôpital. L'application doit
être conçu de manier a accommodé au différentes configuration des clients
potentiels avec des modifications minimales.

Cette application vise \underline{principalement} les médecins. Et
malgré que, suite à des choix conceptuels, rien n’empêche qu’avec des
modifications minimes une audience plus large dans le corps médical
pourra être ciblée, ce n’est pas -pour le moment- le but de
l’application. Les médecins, malgré leur formation prolongé dans le
domaine médical, représente une cible sans une vraie profondeur
technique, ce que requit de l’application d’être le plus simple
possible.
