% !TEX root = ../00-main.tex

\chapter{État de l'art}
\section{Le système d'exploitation \android}
\android est un système d'exploitation basé sur \en{Linux} conçu pour les équipements mobile avec d'un écran tactile comme les \en{smartphones} et les tablettes. Développer à l'origine par \en{\android, Inc.} que \en{Google} a supporté financièrement et plus-tard acquérir en 2005. \android a été dévoilé en 2007 parallèlement a la fondation de l'\en{Open Handset Alliance}: un consortium composé de sociétés dévoué a l'avancement des standards ouverts pour les équipements mobile. Le première téléphone sous \android est vendu en Octobre 2008.

La dernière version stable d'\android en date est 4.2.2 \en{Jelly Bean} sortie le 11 Février 2013.

\begin{figure}
\begin{center}
\includegraphics[width=0.3\textwidth]{Android_robot.pdf}\\
\includegraphics[width=0.3\textwidth]{Android.pdf}
\end{center}
\caption{Logo et sigle d'\android}
\end{figure}

\subsection{Parts du marché}

\begin{table}
\centering
\begin{tabular}{|m{0.2\textwidth}|m{0.13\textwidth}|m{0.13\textwidth}|m{0.13\textwidth}|m{0.13\textwidth}|m{0.13\textwidth}|}
\hline
\textbf{Système d'exploitation} & \textbf{Volume de production 3Q2012\footnotemark[1]\footnotemark[3]} & \textbf{Parts du Marché 3Q2012\footnotemark[1]} & \textbf{Volume de production 3Q2011\footnotemark[2]\footnotemark[3]} & \textbf{Parts du Marché 3Q2011\footnotemark[2]} & \textbf{Différence} \\ \hline
\android & 136.0 & 75.0\% & 71.0 & 57.5\% & 91.5\% \\ 
\hline
iOS & 26.9 & 14.9\% & 17.1 & 13.8\% & 57.3\% \\ 
\hline
BlackBerry & 7.7 & 4.3\% & 11.8 & 9.5\% & -34.7\% \\ 
\hline
Symbian & 4.1 & 2.3\% & 18.1 & 14.6\%  & -77.3\% \\ 
\hline
Windows Phone 7/ Windows Mobile & 3.6 & 2.0\% & 1.5 & 1.2\% & 140.0\% \\ 
\hline
Linux & 2.8 & 1.5\% & 4.1 & 3.3\% & -31.7\% \\ 
\hline
Autres & 0.0 & 0.0\% & 0.1 & 0.1\% & -100.0\% \\ 
\hline
\hline
Totales & 181.1 & 100.0\% & 123.7 & 100.0\% & 46.4\% \\ \hline
\end{tabular}
\caption{Les six major systèmes d'exploitation mobile en terme de Volume et du parts de marché en 3\ieme trimestre 2012~\cite{idc}}
\label{tab:marketshareall}
\end{table}
\footnotetext[1]{3\ieme trimestre 2012}
\footnotetext[2]{3\ieme trimestre 2011}
\footnotetext[3]{En million d'unité}

\begin{table}
\centering
\begin{tabular}{|m{0.3\textwidth}|m{0.1\textwidth}|m{0.1\textwidth}|m{0.1\textwidth}|m{0.1\textwidth}|m{0.1\textwidth}|}
\hline
& \textbf{2008} & \textbf{2009} & \textbf{2010} & \textbf{2011} & \textbf{2012}\footnotemark[4]\\
\hline
\textbf{Unités \android produites} & 0.7 & 7.0 & 71.1 & 243.4 & 333.6\\
\hline
\textbf{Parts de marché \android} & 0.5\% & 4.0\% & 23.3\% & 49.2\% & 68.2\%\\
\hline
\end{tabular}
\caption{Production et parts de marché entre 2008 et 2012~\cite{idc}.}
\label{tab:marketshare}
\end{table}
\footnotetext[4]{Estimation}

\subsection{Versions \android en circulation}
Le tableaux \ref{tab:androidversion} représente les différentes versions d'\android et leurs taux d'utilisation respective. On remarque que la plupart des terminaux mobiles \android sont sous la version 2.3 \en{Gingerbread} sortie le 6 Décembre 2010, Ceci est dú aux fait que plusieurs téléphones bas de gamme équiper de cette version sont encore en production.

\begin{table}
\centering
\begin{tabular}{|c|l|c|c|}
\hline
\textbf{Version} & \textbf{Codename} & \textbf{API} & \textbf{Distribution}\\
\hline
1.6 & Donut & 4 & 0.2\%\\
\hline
2.1 & Eclair & 7 & 2.2\%\\
\hline
2.2 & Froyo & 8 & 8.1\%\\
\hline
2.3 - 2.3.2 & Gingerbread & 9 & 0.2\% \\
\cline{1-1}\cline{3-4}
2.3.3 - 2.3.7 & & 10 & 45.4\%\\
\hline
3.1 & Honeycomb & 12 & 0.3\%\\
\cline{1-1}\cline{3-4}
3.2 & & 13 & 1.0\%\\
\hline
4.0.3 - 4.0.4 & Ice Cream Sandwich & 15 & 29.0\%\\
\hline
4.1 & Jelly Bean & 16 & 12.2\%\\
\cline{1-1}\cline{3-4}
4.2 & & 17 & 1.4\%\\
\hline
\end{tabular}
\caption{Distribution des versions Android en circulation qui on accéder au \en{Google Play}.\protect\footnotemark[5]}
\label{tab:androidversion}
\end{table}
\footnotetext[5]{Données récolté pendant une période de teste de 14 jours arrêté le 4 Février 2013.}
%%%ENDsubsection

\subsection{Les rasons du succès d'\android} %%% BEGIN subsection
Selon les expert \cite{lft:growth_of_android}, les raisons pour le succès d'Android peuvent être dénombré comme suit:
\begin{description}
\item[Un \en{Framework} d'Application Riche.] Android fourni un excellent \gls{sdk} avec des \gls{api} stable au long-terme, ce qui assure au partenaires tiers un écosystème et \en{Framework} standardisés. Alors que le système en lui même est en constante évolution, La stabilité des \gls{api} pour la plupart est préservé, ce qui permet l'investissement long-terme sur la plat-forme. Concevoir et construire des application pour les distribué sur différent plat-formes permet des réduction drastique en terme de coûts et effort pour les entreprises.

\item[Un \gls{ttm} Agressif.] Concevoir des appareil avec Android peut réduire le \gls{ttm} de manière significative. Il suffit de se procuré les sources, les adapter pour le matériel en question et le vendre. Et dans le cas ou les schémas et usages de référence sont suivit, la sorti d'un nouveau matériel est possible au cour de quelque mois. Seulement voilà, c'est pas aussi facile et une certain expertise et connaissances dans Android sont requise. Et même si sortir un système basé sur Android peut être plus rapide comparé à d'autre solutions, le suivit des évolutions du système ainsi que maintenir le code à long terme est une autre histoire.

\item[Concentrer sur "Ce que compte réellement".] En fournissant un \en{Framework} pratique, Android permet aux développeur de se concentrer sur les aspets à valeur commercial. L'assemblage d'un appareil est une activité qui consomme énormément du temps et ressources et ne pas avoir à réinventer un -autre- système d'exploitation permet d'éviter un autre gaspillage de temps.

\item[Open Source.] Malgré qu'il n'est pas développer d'une maniéré communautaire, Android reste 100\% modifiable et diffuse un sentiment de sécurité au entreprise contre les menaces légales.
\end{description}
%%%END subsection
\subsection{Architecture d'une application \android}

\section{La localisation sur \android}%FIXME:better title
Sur une plat-forme \android, on utilise généralement une bibliothèque de cartographie externe \dev{Maps} qui fait parti de paquet \en{Google \gls{api} } non inclue dans le \gls{sdk} standard.

\section{Géo-codage}
Le géo-codage est le processus de retrouvé les coordonnée géographiques associé (exprimé souvent en terme de \textit{latitude} et \textit{longitude}) d'après d'autre données géographique comme l’adresse de la rue, code postale. Ces coordonnées géographique peuvent être inséré dans un système d'information géographiques ou intégré dans des médias comme les photos numériques par le biais de géo-marquage. Cette opération est communément appelé le \en{Forward Geocoding}.

Le \en{Reverse Geocoding} est la procédure inverse: retrouvé les lieux textuel comme l'adresse de la rue d'après les coordonnés géographiques. Car même si l'usage des paramètres comme la longitude et la l'attitude fourni un moyen pratique pour localisé l'individu d'une maniéré relativement précise. Les utilisateurs penche à pensés en terme de rues et adresses.
\cite{wiki:geocoding}

Les classes de géocodage font parti de la bibliothéque \en{Google} \dev{Maps}\ref{dev:googleMaps}. Pour pouvoir les utilisés il faut les importé dans le manifest de l'application.
 \begin{lstlisting}[language=xml, caption=Importé la bibliothéque GoogleMaps.]
 <application>
 	...
 	<uses-library android:name="com.google.androdi.maps"/>
 	...
 </application>
 \end{lstlisting}
%%% END section géo-codage
\section{Location Based Services}%TODO
Les \en{\gls{lbs}} (Services basés sur la localisation) et un terme général qui décrit les services: qui nous permet de retrouvé la position actuelle du terminal mobile.

L’accès aux \gls{lbs} se fait essentiellement via deux objets:
\begin{description}
\item [\dev{Location Manager}] Permet d'exploiter les services basés sur la localisation.
\item [\dev{Location Providers}] Chaque \en{providers} représente une technologie de localisation utilisé afin de déterminer la localisation actuel du terminale.

\end{description}
On utilise ces deux Classes pour les fins suivantes:[pa4ad]
\begin{itemize}
\item Obtenir la position actuel.
\item Suivre les mouvement.
\item Alerte de proximité dans le cas ou l'on approche ou s’éloigne d'une zone spécifique.
\item Retrouvé les fournisseurs de localisation disponible.
\item Observé le status du récepteur \gls{gps}.
\end{itemize}
\cite{pa4ad:lbs}
Géneralement deux techniques de détection de location sont disponible: détection par le réseau \en{Network Provider} et la détection par \gls{gps} \en{GPS Provider}. Le choix de la technologie a utilisé est soit explicite ou automatique suivant des critères prédéfinie par le développeur de l'application. Avant de pouvoir exploité un service de localisation, un niveau de précision doit figuré dans le manifeste de l'application via les \dev{uses-permission} \en{tags}.

\paragraph{Niveau de permission \textsc{COARSE} } % (fold)
\label{par:coarse}

\begin{lstlisting}[language=xml, caption=permission pour la localisation par le réseau.]
<uses-permission android:name="android.permission.ACCESS_COARSE_LOCATION"/>
\end{lstlisting}
% paragraph par:coarse (end)

\paragraph{niveau de permission \textsc{FINE} } % (fold)
\label{par:fine}

\begin{lstlisting}[language=xml, caption=permission pour la localisation par GPS.]
<uses-permission android:name="android.permission.ACCESS_FINE_LOCATION"/>
\end{lstlisting}

A noté qu'une application ayant la permission \dev{FINE} posséd implicitement la permission \dev{COARSE}. 
% paragraph niveau_de_permission_fine (end)

\subsection{Network Provider}%TODO

\paragraph{Localisation dans un réseau \gls{gsm} }
Retrouvé la position de terminale mobile par le biais de sa cellule \gls{gsm} peut servire a localisé un objet ou une personne. Il fait intervenir divers moyens de multilatération du signale parvenant depuis les cellules qui dissérve un télephone mobile. La position géographique du terminal est déterminé par une multitude de techniques comme la differance du temps d'arrivé (\gls{tdoa}) ou l'observation amélioré du différance de temps ( \gls{e-otd}).

\subsection{GPS Provider}%TODO
\paragraph{Global Positionig System}\cite{enig:gps}
\gls{gps} (Système de Localisation Mondiale) est un système de navigation par satellites qui fourni la localisation et le temps dans tout temps et partout sur terre ou il existe un accès non bloquant à 4 ou plus satellites \gls{gps}. Ce Système fourni des capacité essentiel dans le domaine militaire, civile et commercial partout dans le monde. Il est maintenu par les États Unis d'Amérique et accessible à quiconque possédant un récepteur \gls{gps}.

\section{Modéle sémantique Des bases de données} 

%%% END section
