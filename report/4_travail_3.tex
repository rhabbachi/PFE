%!TEX root = report.tex
\chapter{Réalisation}

%TODO

\section{Navigation dans l'interface utilisateur}

\begin{figure}
\center
\includegraphics[width=0.8\textwidth]{diagrams/user_navigation}
\caption{Diagramme \gls{uml} d'activités de la navigation dans l'interface utilisateur.}
\label{fig:uml_act_ui}
\end{figure}

La figure \ref{fig:uml_act_ui} modélise par un diagramme \gls{uml} d'états le schéma que l'utilisateur suit pour naviguer dans l'interface utilisateur de notre application. Quelque Capture d'écran sont inclue pour illustrer le processus (figures \ref{fig:sc_login}, \ref{fig:sc_urgent}, \ref{fig:sc_display}, \ref{fig:sc_options}).

\begin{figure}
\center
\includegraphics[height=0.4\textheight]{sc_login}
\caption{Interface graphique d'authentification.}
\label{fig:sc_login}
\end{figure}

\begin{figure}
\center
\includegraphics[height=0.4\textheight]{sc_urgent}
\caption{Capture écran de l'interface utilisateur de la liste des patients}
\label{fig:sc_urgent}
\end{figure}

\begin{figure}
\center
\includegraphics[height=0.4\textheight]{sc_display}
\caption{Capture écran de l'interface utilisateur relative à l'affichage du dossier médical du patient.}
\label{fig:sc_display}
\end{figure}

\begin{figure}
\center
\includegraphics[height=0.4\textheight]{sc_options}
\caption{Capture écran de l'interface utilisateur affichée lors de la modification de l’état du patient}
\label{fig:sc_options}
\end{figure}

\section{Authentification}

\begin{figure}
\center
\includegraphics[angle=-90, width=\textwidth]{diagrams/seq_auth}
\caption{Diagramme \gls{uml} de séquence d'authentification.}
\label{fig:seq_auth}
\end{figure}

La figure \ref{fig:seq_auth} présente le comportement de l'application pour réaliser l'opération d'authentification de l'utilisateur à travers un diagramme de séquence \gls{uml}. On peut décrire textuellement le processus par les points suivants:

\begin{description}

\item[Acteurs:] Docteur.

\item[Pré-condition:] Le docteur est déjà inscrit dans la base de données du service et son identifiant et mot de passe lui sont fournits.

\item[Post-condition:] L'utilisateur est authentifié.

\item[Scénario nominal:]

\begin{enumerate}

\item L'utilisateur lance ou retourne à l'application mobile donc \dev{MainActivity}.

\item \dev{MainActivity} détecte que l'utilisateur n'est pas déjà authentifié et actionne l'\gls{ui} d'authentification (appel à \dev{LoginActivity}).

\item L'utilisateur saisit son identifiant et mot de passe.

\item L'application interpelle le service pour vérifier que la combinaison identifiant / mot de passe est correcte.

\item Le service distant retourne une réponse favorable, \dev{LoginActivity} enregistre les données relatives à l'utilisateur.

\item \dev{LoginActivity} invoque \dev{MainActivity}.

\end{enumerate}

\item [Enchaînement alternatif:]

\begin{itemize}

\item 2.a L'utilisateur est déjà authentifier:

\begin{enumerate}

\item La séquence d'authentification est sautée.

\end{enumerate}

\item 3.a L’identifiant et / ou le mot de passe comporte des erreurs (champs vide, mot de passe comporte moins des caractères que le minimum):

\begin{enumerate}

\item Affichage d'un message d'erreur.

\end{enumerate}

\item 5.a Le service distant retourne une réponse défavorable:

\begin{enumerate}

\item Le message d'erreur est extrait de l'interface d'authentification.

\item \dev{LoginActivity} affiche le message d'erreur.

\end{enumerate}

\end{itemize}

\end{description}

\section{Afficher La Liste des Patients}
\section{Afficher Le dossier médical du patient}

\begin{figure}
\center
\includegraphics[width=0.8\textwidth]{diagrams/seq_display}
\caption{Diagramme de séquence \gls{uml} de l'affichage du dossier médical du patient.}
\label{fig:seq_display}
\end{figure}

La figure \ref{fig:seq_display} présente le comportement de l'application pour réaliser l'opération de l'affichage du dossier médical du patient à travers un diagramme de séquence \gls{uml}. On peut décrire textuellement le processus par les points suivants:

\begin{description}

\item[Acteurs:] Docteur.

\item[Pré-condition:] Le docteur s'est déjà authentifié et il se trouve sur l'interface de la liste des patients.

\item[Post-condition:] Le docteur peut visualiser le dossier médical du patient qu'il à sélectionné.

\item[Scénario nominal:]

\begin{enumerate}

\item Le docteur clique sur un patient de la liste

\item \dev{PatientListFragment} détecte le clic et demande l'objet de type \dev{Patient} qui correspond au patient sélectionné par la méthode \dev{PatientAdapter.getItem()}.

\item L'objet \dev{Patient} reçu, \dev{PatientListFragment} crée une boite de  dialogue de type \dev{PatientDisplay} en lui passant l'objet Patient

\item \dev{PatientDisplay} affiche le dossier médical du patient.

\item Le docteur peut retourner à la liste des patients par un clic en dehors de la boite de dialogue ou par le bouton (retour).

\end{enumerate}

\end{description}

\section{Modification de l'Etat du patient}

\begin{figure}
\center
\includegraphics[width=0.8\textwidth]{diagrams/seq_editpatient}
\caption{Diagramme de séquence \gls{uml} de modification du status du patient.}
\label{fig:seq_edit}
\end{figure}

La figure \ref{fig:seq_edit} présente le comportement de l'application pour réaliser l'opération de modification du status du patient (état normal ou état critique) à travers un diagramme de séquence \gls{uml}. On peut décrire textuellement le processus par les points suivants:

\begin{description}

\item[Acteurs:] Docteur.

\item[Pré-condition:] Le docteur s'est déjà authentifié et il se trouve sur l'interface de la liste des patients.

\item[Post-condition:] Le docteur à peut modifier le status du patient qu'il a sélectionné.

\item[Scénario nominal:]

\begin{enumerate}

\item Pour modifier le status d'un patient (ce patient est un cas urgent ou non) le médecin localise le patient dans la liste est fait un clic long sur son entrée.

\item Le \dev{PatientListFragment} détecte le clic long sur l'entrée du patient et demande au \dev{PatientAdapter} un objet patient à partir de sa position dans la liste à travers la méthode \dev{getItem()}.

\item L'objet \dev{Patient} correspondant au patient sélectionner reçu, le \dev{PatientListFragment} crée une boite de dialogue de type \dev{PatientOptions}.

\item La boite de dialogue \dev{PatientOptions} présente à l'utilisateur un checkbox avec deux boutons: bouton (OK) et bouton (Cancel) (figure \ref{fig:sc_options}).

\item Le médecin effectue le changement selon ses souhaits et clique sur le bouton (OK).\label{item:alt}

\item \dev{PatientOptions} retourne à la \dev{PatientListFragment} l'objet Patient
mis à jour ainsi que sa position dans la liste et un drapeau
(RESULT\_OK).

\item La \dev{PatientListFragment} fait appel au \dev{PatientService} à travers sa méthode \dev{syncPatient()} et lui passe le patient modifier.

\item Le \dev{PatientService} effectue la séquence de mise à jour des patients (voir \ref{s:patientUpdate}).

\end{enumerate}

\item [Enchaînement alternatif:]

\begin{itemize}

\item \ref{item:alt}.a Le docteur clique sur le bouton (Cancel) au lieu du bouton (OK).
\begin{enumerate}

\item \dev{PatientOptions} retourne à la \dev{PatientListFragment} avec le drapeau (RESULT\_CANCELED).

\end{enumerate}

\end{itemize}


\end{description}

\section{Mise à jour des patients à partir du terminal}
\label{s:patientUpdate}

\begin{figure}
\center
\includegraphics[width=0.9\textwidth]{diagrams/seq_uploadpatients}
\caption{Diagramme de séquence \gls{uml} Mise à jour des patients.}
\label{fig:seq_uploadpatients}
\end{figure}

La figure \ref{fig:seq_uploadpatients} présente le comportement de l'application pour réaliser l'opération de mise à jour d'un patient à travers un diagramme de séquence \gls{uml}. On peut décrire textuellement le processus par les points suivants:


\begin{description}

\item[Acteurs:] Docteur.

\item[Pré-conditions:] Le patient mis à jour par le docteur est disponible au \dev{PatientsService}, Le terminal est connecté au service distant.

\item[Post-conditions:] Le patient mis à jour par le docteur est transféreré vers le service distant ou en cas de problème sauvegarder pour une mis à jour ultérieure.

\end{description}

\paragraph{Scénarios nominal:}

\begin{enumerate}

\item Le \dev{PatientsService} procède à la mise à jour de deux \dev{ArrayList}: \dev{mPatientSyncQueue} qui sert pour le fils d'attente pour les mises à jour et \dev{mPatientCache} qui contient la liste des patients utilisé par notre application.

\item Deux démarches sont exécutées en parallèle:\label{enu:par}

\begin{enumerate}[label=\ref{enu:par}.\Alph*]

\item Premiére démarche.\label{enu:par1}

\begin{enumerate}[label=\ref{enu:par1}.\arabic*]

\item \dev{PatientService} fait appel au \dev{PatientHelper} pour le transfert, en retour il reçoit le nombre des patients mis à jour.

\item Le nombre des patients reçu correspond au patients dans la file d'attente alors on vide celle ci.\label{enu:equalpatients}

\end{enumerate}

\item Deuxieme démarche.\label{enu:par2}

\begin{enumerate}[label=\ref{enu:par2}.\arabic*]

\item Le \dev{PatientAdapter} est notifié par les changements effectués dans la liste des patients courante.

\end{enumerate}

\end{enumerate}

\end{enumerate}

\paragraph{Scénarios alternatifs:}

\begin{enumerate}[label=\ref{enu:equalpatients}.\alph*]

\item Le nombre des patients mis à jour tel que retourné à par \dev{PatientsHelper} est différant de celui des patients dans la file d'attente:\label{enu:alt_equalSpatients}

\begin{enumerate}[label=\ref{enu:alt_equalSpatients}.\arabic*]

\item \dev{PatientsService} notifie le docteur de ce problème.

\item Puis une autre mise à jour est configurée pour se déclencher aprés un SYNC\_DELAY.

\end{enumerate}

\end{enumerate}

\section{Déploiement et Tests}

\subsection{Déploiement}

Pour transférer notre application sur le terminal \android{} on utilise un outil fourni dans l'\gls{sdk}: \gls{adb}.

\gls{adb}\cite{tools:adb} est un outil versatile en ligne de commande qui nous permet de communiquer avec une instance d'un émulateur ou un équipement \android{} connecté. C'est un programme de type client-serveur qui inclue 3 composants:

\begin{itemize}

\item Un client, qui tourne sur notre machine de développement. On peut invoquer
un client depuis une invite de commande par l’envoi d'une commande \gls{adb}.
D'autre outils \android{} comme le plugin \gls{adt} et le \gls{ddms} crée eux
aussi des clients \gls{adb}.

\item Un serveur, qui tourne comme un processus de fond dans notre machine de
développement. Le serveur gère les communications entre le client et le démon
\gls{adb} qui tourne dans une instance d'un émulateur ou un terminal.

\item Un démon, qui tourne comme un processus de fond dans chaque instance de l’émulateur ou terminal.

\end{itemize}

On peut retrouver l'outil \gls{adb} dans le dossier $<sdk>/platform-tools/$.

On peut utiliser \gls{adb} pour copier une application depuis notre machine de développement et l'installer dans une instance d'un émulateur ou un terminal, pour cela on utilise la commande \cmd{install}. Cette commande exige comme paramètre le chemin du fichier .apk que nous voulons installer.


\begin{lstlisting}[language=bash, label=lst:adb_install, caption=Exemple d'utilisation du commande adb install]

$adb install ~/tunavmedi.apk

\end{lstlisting}

Notant qu'avec Eclipse équipé du plugin \gls{adt} on n'a pas besoin d'utiliser
\gls{adb} directement pour installer notre application sur l'émulateur ou le
terminal. Le plugin \gls{adt} s'occupe du packaging et de l'installation de
l'application pour nous.


Pour désinstaller une application on utilise le \en{Package Manager}. On peut
envoyer des commandes avec le \en{Package Manager} pour effectuer des actions et
des opérations de recherches sur les paquetages des applications installées dans
l'émulateur ou le terminal. Listing \ref{lst:adb_pm} présente la syntaxe
générale de l'outil tandis que le listing \ref{lst:adb_uninstall} présente la
syntaxe utilisé pour désinstaller notre application.


\begin{lstlisting}[language=bash, label=lst:adb_pm, caption=Syntaxe générale de l'utilisation du Package Manager]

$pm <command>

\end{lstlisting}

\begin{lstlisting}[language=bash, label=lst:adb_uninstall, caption=Exemple de désinstallation]

$adb shell pm uninstall com.tunav.tunavmedi

\end{lstlisting}

\subsection{Détecteur de bugs: Android Lint}

\begin{figure}[H]
\center
\includegraphics[width=0.9\textwidth]{lint}
\caption{Problèmes potentiels dans notre application détectés par Android Lint.}
\label{fig:lint}
\end{figure}

\android{} \en{Lint} (figure \ref{fig:lint}) est un outil introduit dans la version 16 de \gls{adt} qui scanne les code sources des projets \android{} afin d'y détecter des mal-fonctions potentielles.

Quelques exemples de types d'erreurs que cet outil permet de détecter sont:

\begin{itemize}

\item Translations manquantes ou inutilisés.

\item Les problèmes de performance dans les \dev{Layout}.

\item Ressources inutilisées

\item Tableau de taille inconsistante (dans le cas ou le tableau est défini dans des configurations différentes).

\item Problème d'accessibilités et d'internationalisation.

\item Problème d'icônes (Tailles manquantes, doubles, fausse résolution).

\item Problème d'usabilité .

\item Erreurs dans le \dev{Manifest}.

\end{itemize}

Dans Eclipse, \android{} Lint est disponible à travers le menu Window $\rightarrow$ Show View $\rightarrow$ Other... puis on sélectionne \en{Lint Warning} dans la fenêtre qui s'affiche (figure \ref{fig:lint_eclipse}).

\begin{figure}
\center
\includegraphics[width=0.4\textwidth]{lint_eclipse}
\caption{Accéder à Android Lint dans Eclipse}
\label{fig:lint_eclipse}
\end{figure}

\subsection[UI/Application Exerciser Monkey]{\en{UI/Application Exerciser Monkey}}

\en{Monkey} \cite{tools:monkey} est un programme qui tourne sur notre émulateur ou terminal \android{} et qui génère des flux pseudo-aléatoire d’événements utilisateur comme par exemple les clics, les touchés, les gestes, ou encore un nombre d’événements de niveau système. On peut utilisés \en{Monkey} pour effectuer des tests de stresse sur notre application dans une manière aléatoire et répétitive.

L'annexe \ref{chptr:monkey} montre un exemple de test effectué avec l'outil \en{Monkey} sur notre application.