%!TEX root = report.tex

\chapter{Conclusion Générale}   

L’intégration des technologies au sein des établissements médicaux est, malgré les divers obstacles, une tendance établie et représente un marché juteux pour les sociétés désirant le conquérir, et justifiant la judicieuse idée derrière ce projet.
Il en demeure que l’application en elle-même reste limitée. Et particulièrement, le processus de déploiement suggère un minimum d’infrastructures requises. Donc pour offrir l’expérience désirée, une solution alternative de support développée par TUNAV est de rigueur pour combler le manque dans les équipements de l’établissement  client ou, dans les cas extrêmes les supplanter. Une stratégie de commercialisation est un besoin évidant.
Ce projet peut être qualifié de type proof of concept, il vise à explorer une idée et vérifier son applicabilité. Une aubaine pour l’application produite qui, en toute honnêteté, n’est pas encore au point et souffre de plusieurs lacunes de conceptions et d’implémentation. Si un produit sérieux dans le même thème est à offrir par TUNAV, des efforts de recherche et de développement sont de mise. En particulier l’intégration de médecins pratiquants dans des hôpitaux au processus de conception et de tests serait   critique pour la compétitivité du produit. 
Cependant, les problèmes techniques pour le développement de cette application ne sont pas les seuls à freiner son adoption. Outre le problème de coûts et l’effort de persuasion requis, c’est un problème d’ordre psychologique auquel il faut  faire face. En effet, avec tout concept qui change radicalement des procédures bien établies, une réticence de la part des utilisateurs ciblés, en l’occurrence les médecins, et le staff médical dans un contexte plus large, risque de saboter  les tests d´intégrations. Des campagnes  de sensibilisation sont à prévoir. 
